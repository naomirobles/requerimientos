\documentclass[11pt]{article}  

%%%%%%%%%%%%%% PREÁMBULO %%%%%%%%%%%%
\title{Propuesta de Proyecto}
\usepackage[spanish]{babel}
\usepackage[utf8]{inputenc}
\usepackage[T1]{fontenc}     % fuentes con soporte de acentos mejorado
\usepackage{listingsutf8}
\usepackage{textcomp}      % para algunos símbolos de texto
\usepackage{upquote}    
\usepackage{amsmath}
\usepackage{amssymb} 
\usepackage{graphicx}
\usepackage{color}
\usepackage{enumitem}
\usepackage{subfigure} 
\usepackage{float} 
\usepackage{capt-of} 
\usepackage{sidecap} 
\sidecaptionvpos{figure}{c} 
\usepackage{caption} 
\usepackage{commath}
\usepackage{graphicx, amsmath, amsthm, latexsym, amssymb, amsfonts, epsfig, float, enumerate, color, listings, graphicx, fancyhdr}
\usepackage{cancel} 
\usepackage[table,xcdraw]{xcolor}
\usepackage{anysize} 
\usepackage{listings}
\usepackage{xcolor}  % Para colores
\usepackage{booktabs}
\usepackage{pgfgantt}
\usepackage{pdflscape}

\marginsize{2cm}{2cm}{2cm}{2cm} % Izquierda, derecha, arriba, abajo
\usepackage{appendix}
\usepackage[style=ieee, backend=biber]{biblatex}
\renewcommand{\appendixname}{Apéndices}
\renewcommand{\appendixtocname}{Apéndices}
\renewcommand{\appendixpagename}{Apéndices} 
\usepackage[colorlinks=true,plainpages=true,citecolor=blue,linkcolor=black]{hyperref}
\usepackage{fancyhdr} 
\usepackage{url}
\pagestyle{fancy}
\fancyhf{}
\fancyhead[L]{\footnotesize Ingeniería de Software para Sistemas Inteligentes} 
\fancyfoot[C]{\thepage} 
\renewcommand{\footrulewidth}{0.4pt}
\usepackage[utf8]{inputenc}

\definecolor{dkgreen}{rgb}{0,0.6,0} 
\definecolor{gray}{rgb}{0.5,0.5,0.5} 
\definecolor{verde}{rgb}{0.18,0.549,0.2}
\definecolor{rojo}{rgb}{1, 0.302, 0.431}
\definecolor{black}{rgb}{0, 0, 0}
\definecolor{blue}{rgb}{0, 0.4648, 0.71}
\definecolor{charcoal}{RGB}{54,69,79}

\definecolor{codegreen}{rgb}{0,0.6,0}          % Verde para comentarios
\definecolor{codegray}{rgb}{0.5,0.5,0.5}       % Gris para números de línea
\definecolor{codepurple}{rgb}{0.58,0,0.82}     % Púrpura para strings
\definecolor{backcolour}{rgb}{0.98,0.98,0.99}  % Color de fondo
\definecolor{keywordcolor}{rgb}{0,0.59,0.61}   % Color para palabras clave (class, def, etc.)
\definecolor{titlecolor}{rgb}{0.53,0,0}        % Color para títulos de clases y funciones
\definecolor{stringcolor}{rgb}{0,0.36,0.37}    % Color para strings
\definecolor{commentcolor}{rgb}{0.58,0.65,0.65} % Color para comentarios

% Definir el estilo para el código
\lstdefinestyle{mystyle}{
    backgroundcolor=\color{backcolour},   
    commentstyle=\color{commentcolor},
    keywordstyle=\color{keywordcolor},
    numberstyle=\tiny\color{codegray},
    stringstyle=\color{stringcolor},
    basicstyle=\ttfamily\footnotesize,
    breakatwhitespace=false,         
    breaklines=true,                 
    captionpos=b,                    
    keepspaces=true,                 
    numbers=left,                    
    numbersep=5pt,                  
    showspaces=false,                
    showstringspaces=false,
    showtabs=false,                  
    tabsize=2,
    emph={class,def,__init__,return,import}, % Palabras clave adicionales
    emphstyle=\color{titlecolor},            % Estilo para palabras clave adicionales
    morekeywords={class,def,__init__,return,import}, % Añadir palabras clave
     % ahora añadimos los símbolos faltantes
}

\lstset{style=mystyle}
\renewcommand{\arraystretch}{1.3} % Aumenta el espacio entre filas en tablas

% --- Archivo de bibliografía -----------------
\addbibresource{repbib.bib}


\title{Requerimientos Funcionales y No Funcionales}

%%%%%%%%%%%%%%%%%%%% TERMINA PREÁMBULO %%%%%%%%%%%%

\begin{document}

%%%%%%%%%%%%%%%%%%%% PORTADA %%%%%%%%%%%%%%%%%%%%%%%

\begin{titlepage}
  \thispagestyle{empty}
  \pagecolor{white}
  
  % Encabezado con logos
  \begin{center}
    \vspace{1cm}
    
    % Logos en la parte superior
   
      \centering
          \vspace{2cm}
      \includegraphics[height=2.35cm]{img/logos.png}
    
    \vspace{3cm}
    
    % Tipo de trabajo
    {\normalsize\color{charcoal}\textsc{Tarea}}\\[0.5cm]
    
    % Título principal
    {\huge\bfseries\color{black}Documento de especificación de requerimientos de software}\\[0.5cm]

    
    \vspace{1.5cm}
    
    % Información de autores
    {\small\color{charcoal}\textsc{Realizada por}}\\[0.3cm]
    
    {\normalsize\color{black}Robles Guzmán Naomi Isabel}\\[0.2cm]
    {\normalsize\color{black}Ugalde Téllez Aarón}\\[0.5cm]

    
    \vspace{0.8cm}
    
    % Información académica
    {\small\color{charcoal}\textsc{Para la materia de}}\\[0.2cm]
    {\normalsize\color{blue}Ingeniería de Software para Sistemas Inteligentes}\\[0.5cm]
    
    {\small\color{charcoal}\textsc{Impartida por}}\\[0.2cm]
    {\normalsize\color{black}Chadwick Carreto Arellano}\\[0.5cm]
    
    {\small \color{charcoal}\textsc{Grupo}}\\[0.3cm]
    {\normalsize\color{black}6BM1}\\[0.8cm]
    
    \vfill
    
    % Línea decorativa y fecha
    {\color{blue}\rule{0.4\textwidth}{1pt}}\\[0.3cm]
    {\large\color{charcoal}19 de septiembre de 2025}
    
    \vspace{1cm}
    
  \end{center}
\end{titlepage}                                                     

%%%%%%%%%%%%%%%%%%%% ÍNDICES %%%%%%%%%%%%%%%%%%%%%%%%%%
\tableofcontents 
\pagebreak


%%%%%%%%%%%%%%%%% CUERPO %%%%%%%%%%%%%%%%%%%%%%%%

\section{Introducción}
Él presente \textit{software} plantea una aplicación web responsiva de gestión de tareas académicas y profesionales que integra
 un motor de recomendaciones educativas basado en análisis de palabras clave. Ofreciendo una conexión directa hentre recursos educativos y 
 gestión de actividades, alineando planificación y ejecución en un mismo entorno.\\\\
 Los requisitos de \textit{software} establecen las especificaciones detalladas que definen que debe hacer un sistema, cómo debe 
 comportarse y bajo que condiciones debe operar \autocite{Requirements2024}. Este documento  tiene como propósito establecer de
 manera precisa y completa todas las funcionalidades, características y restricciones que deberá cumplir el sistema de gestión de
 tareas académicas y profesionales, garantizando así el desarrollo de una solución que satisfaga plenamente las necesidades identificadas.
 \\\\
 
\vspace{0.7cm}
\section{Requerimientos Funcionales}
\begin{table}[H]
    \centering
    \begin{tabular}{|p{2cm}|p{13cm}|}
    \hline
    \multicolumn{2}{|c|}{\textbf{1. Cuenta e Inicio de Sesión}} \\
    \hline
    \textbf{Clave} & \textbf{Descripción } \\
    \hline
    RF1.1 & El sistema permite al usuario crear una cuenta ingresando correo electrónico, contraseña,
     país de origen y último grado de estudios. \\
     \hline
    RF1.2 & El sistema permite al usuario iniciar sesión con su correo electrónico y contraseña registradas. \\
    \hline
    RF1.3 & El sistema permite al usuario recuperar su contraseña mediante un correo electrónico de recuperación.\\
    \hline
    RF1.4 & El sistema permite al usuario ver y actualizar su perfil, incluyendo nombre, foto de perfil, país de origen y último grado de estudios. \\
    \hline
    RF1.5 & El sistema permite al usuario terminar su sesión.\\
    \hline
    \multicolumn{2}{|c|}{\textbf{2. Registro de Tareas}} \\
    \hline
    RF2.1 & Dado un usuario autenticado, al dar cloc en "Nueva tarea", el sistema abre un modal donde el usuario puede ingresar el título, descripción
     y fecha de vencimiento de la tarea. \\
    \hline
    RF2.2 & Tras ingresar los datos de la tarea, al dar clic en "Guardar", el sistema almacena la tarea en la base de datos con la etiqueta de "Todo".\\
    \hline
    RF2.3 & Al registrar una tarea además de especificar la fecha límite de la tarea, el usuario puede especificar el plazo en el que desea realizarla, es decir,
     los días en el que desea trabajar en la tarea. \\
    \hline
    RF2.4 & Dado un usuario con tareas registradas en el mes, cuando abre la vista mensual, entonces cada día muestra una etiqueta/contador con las tareas 
    y al hacer clic se ve el detalle y acceso para editar. \\
    \hline
    RF2.5 & El sistema permite al usuario modificar tareas. \\
    \hline
    RF2.6 & El sistema permite al usuario eliminar tareas. \\
    \hline
    RF2.7 & Al registrar o editar una tarea el usuario puede crear una categoría para la misma.\\
    \hline
    RF2.8 & El sistema permite al usuario ver las tareas organizadas por categorías. \\
    \hline
    RF2.9 & Al registrar o editar una tarea el usuario puede elegir una categoría existente para clasificarla.\\
    \hline
    RF2.10 & Las tareas con plazo se mostrarán como una etiqueta sobre la vista de calendario en los días especificados.\\
    \hline
    RF2.11 & El sistema permite al usuario marcar tareas como completadas. \\
    \hline
    RF2.12 & Una tarea que haya sido marcada como completada, se vera en la vista de calendario con una línea horizontal sobre el título de la tarea.\\
    \hline
    \multicolumn{2}{|c|}{\textbf{3. Recomendación de recursos para consulta}} \\
    \hline
    RF3.1 & Al agregar una tarea, el sistema genera recomendaciones de consulta (enlaces a artículos, video o páginas web)  \\
    \hline
    RF3.2 & El sistema genera recomendaciones con base al titulo de la tarea y al grado académico que el usuario ingresó al crear la cuenta.\\   
    \hline
    
    \end{tabular}

    \label{tab:wumpus}
\end{table}
%--------------
\section{Requerimientos No Funcionales}
\begin{table}[H]
    \centering
    \begin{tabular}{|p{3cm}|p{12cm}|}
    \hline
    \multicolumn{2}{|c|}{\textbf{Requerimientos No Funcionales (RNF)}} \\
    \hline
    \textbf{Categoría} & \textbf{Descripción} \\
    \hline
    Rendimiento & \textbf{RNF1} Las operaciones de interacción básica en la interfaz (abrir modal, guardar/editar tarea, navegar entre vistas) deberán responder en menos de \textbf{300 ms} en el 95\% de las peticiones bajo carga nominal. \\ 
    \hline
    Disponibilidad & \textbf{RNF2} El sistema deberá garantizar una disponibilidad mínima del \textbf{99.5\%} mensualmente (excluyendo ventanas de mantenimiento programado). \\ 
    \hline
    Seguridad & \textbf{RNF3} Autenticación segura (contraseñas almacenadas con hashing fuerte: bcrypt/argon2). \newline
                \textbf{RNF3} Todas las comunicaciones entre cliente y servidor deben estar cifradas mediante protocolo HTTPS. \newline
                \textbf{RNF3} Los usuarios deben autenticarse mediante credenciales seguras y el sistema debe implementar bloqueo tras 3 intentos fallidos. \\ 
    \hline
    Usabilidad & \textbf{RNF4} Interfaz intuitiva y consistente; realizar las principales acciones (crear tarea, marcar completada, editar) en \textbf{máximo 3 clics}. \newline
                \textbf{RNF4} Soporte móvil (diseño mobile-first) y escritorio. \\ 
    \hline
    Compatibilidad & \textbf{RNF5} Funcionamiento garantizado en las dos últimas versiones de navegadores modernos: Chrome, Firefox, Edge y Safari; versión adaptada para navegadores móviles equivalentes. \\ 
    \hline
    Mantenibilidad & \textbf{RNF6} Código modular y documentado.\\ 
    \hline
    Respaldo y Recuperación & \textbf{RNF7} Copias de seguridad automáticas de la base de datos con retención mínima de \textbf{30 días}. \newline
                             \textbf{RNF7} Objetivos: \textbf{RTO (tiempo de recuperación)} < 4 horas y \textbf{RPO (pérdida de datos máxima)} < 1 hora. \\ 
    \hline
    Recomendaciones & \textbf{RNF8} Las recomendaciones generadas al crear una tarea deberán presentarse en pantalla en menos de \textbf{3 segundos}.\\ 
    \hline
    Legal y Ética & \textbf{RNF9} Incluir términos y condiciones y política de privacidad. \newline
                   \textbf{RNF9} Las recomendaciones deben evitar contenidos que violen derechos de autor o políticas de contenido inapropiado. \\ 
    \hline
    \end{tabular}
    \caption{Requerimientos no funcionales principales}
    \label{tab:nfr}
\end{table}

\section{Historias de Usuario}

%%%%%%%%%% BIBLIOGRAFÍA %%%%%%%%%%%%%%%%%%%%%
\pagebreak
\printbibliography[heading=bibintoc]

\end{document}
